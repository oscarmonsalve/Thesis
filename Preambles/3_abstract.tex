{ % Must have this curly bracket otherwise the whole Section 1 will be left and right indented.

% Indent both left and right margins of abstract by 0.5 in.
\leftskip0.5in\relax
\rightskip0.5in\relax

% APA7 Rule 2.9 Abstract
% Abstracts in paragraph format are written as a single paragraph without indentation of the first line.

% APA7 Rule 2.24 Paragraph indentation
% Exception dot point 4: The first line of the abstract should be flush left.

\noindent Single paragraph. No indentation. Maximum 250 words. Phasellus egestas purus et sem porta venenatis. Vestibulum ante ipsum primis in faucibus orci luctus et ultrices posuere cubilia curae; Sed pulvinar leo ut posuere sollicitudin. Integer sem tortor, tristique id nisl sed, aliquam pharetra tortor. Integer tortor purus, facilisis mollis tortor nec, sodales bibendum urna. Nulla quam enim, feugiat semper neque ut, sagittis ullamcorper urna. Donec et iaculis mi. Etiam congue tellus ut felis viverra, ac mollis eros venenatis. Vivamus eget ante at eros pulvinar tempor nec at arcu. Donec et vestibulum nunc. Fusce sed metus nisi. In leo turpis, mattis a tincidunt luctus, pellentesque et velit. Ut tortor mi, rutrum nec fringilla vitae, maximus et ligula. Vestibulum finibus semper ornare.

% APA7 Rule 2.10 Keywords
% Write the label "Keywords:" (in italic) one line below the abstract, indented 0.5 in. like a regular paragraph, followed by the keywords in lowercase (but capitalize proper nouns), separated by commas. The key words can be listed in any order. Do not use a period or other punctuation after the last keyword. If the keywords run onto a second line, the second line is not indented.

\textit{Keywords:} keyword 1, keyword 2, keyword 3

% Leave an empty line behind
\vspace{0.75\baselineskip}
}